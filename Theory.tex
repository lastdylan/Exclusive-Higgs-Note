%\clearpage
\section{Theory}
\label{sec:theory}

\par The Feynman diagram shown in Figure~\ref{fig:exclH} describes the production
of the exclusive double diffractive Higgs boson. In this QCD mechanism two gluons 
interact through a top quark loop to form the Higgs boson
 while a third gluon is exchanged to keep the protons color neutral and hence intact.  
Calculations for this process have been studied extensively. The most popular 
calculations are performed by the KMR and the CHIDe models, which differ very slightly 
and predict similar cross sections~\cite{Khoze}\cite{Cudell}. 
Both models introduce the proton impact factor through the skewed 
unintegrated gluon density~\cite{Martin}, calculate the probability of additional
gluon density using the Sudakov form factor~\cite{Dokshitzer1980269}.
Both models also take into account the probability of soft interaction between the 
outgoing protons. This interaction tends to destroy the intactness of the
outgoing protons, and hence destroys the rapidity gaps associated with 
exclusive processes. There are two main differences between the KMR and the CHIDe models:
 CHIDe uses different limits to compute the Sudakov factor and also includes
an additional K-factor to introduce some NLO corrections. 
As a result, KMR predicts a higher cross section for the exclusive 
double diffractive Higgs than CHIDe. For a 125\GeV\ Higgs KMR predicts
3 fb at $\sqrt{s}=8$\TeV~\cite{Khoze}. 

\begin{figure}[!h]
\centering
\begin{fmffile}{fgraphs}
\begin{fmfgraph*}(200,200)
%bottom and top verticies
  \fmfleft{P1,P2}
  \fmfright{P1',H,P2'}
%incoming protons to gluon vertices
  \fmf{fermion,tension=2,lab=$P_1$}{P1,g1}
  \fmf{fermion,tension=2,lab=$P_2$}{P2,g2}
%blobs at gluon vertices, 0.16w is the size of blob
\fmfblob{.16w}{g1,g2}
\fmf{gluon}{g1,v1}
\fmf{gluon}{g1,g2}
\fmf{gluon}{v2,g2}
%quark loop was here
\fmf{fermion, tension=1, lab.side=left,lab=$t$}{t,v1}
\fmf{fermion, tension=1.2, lab.side=left,lab=$t$}{v1,v2}
\fmf{fermion, tension=1,lab.side=left,lab=$t$}{v2,t}
\fmf{dashes, lab=$H$}{t,H}
%%gluon from P1 to loop
%\fmf{gluon}{g1,K1}
%%gluon from P2 to loop
%\fmf{gluon}{g2,K2}
%%quark loop was here
%\fmf{fermion, lab.side=right,lab=$b$}{K3,H}
%outgoing protons
\fmf{fermion}{g1,P1'}
\fmf{fermion}{g2,P2'}
%freeze everything in place
%\fmffreeze
%\renewcommand{\P}[3]{\fmfi{plain}{%
%    vpath(__#1,__#2) shifted (thick*(#3))}}
%%lines on P1
%\P{P1}{g1}{2,-1}
%\P{P1}{g1}{-2,1}
%%lines on p2
%\P{P2}{g2}{2,1}
%\P{P2}{g2}{-2,-1}
%%lines on P1'
%\P{g1}{P1'}{-2,-1}
%\P{g1}{P1'}{2,1}
%%lines on P2'
%\P{g2}{P2'}{-2,-1}
%\P{g2}{P2'}{2,1}
\end{fmfgraph*}
\end{fmffile}
\caption{}
\label{fig:exclH}
\end{figure}

\par There are two main sources of theoretical systematic uncertainties
that have to be studied: upper and lower Sudakov form factor scales, unintegrated
gluon densities. Uncertainties associated with the 
rapidity gap survival probability were originally studied for the 
Tevatron and subsequently for the LHC~\cite{Khoze2000wk}. 
In this analysis we will not use rapidity gaps for event selection so 
uncertainties from rapidity gaps are estimated to be minimal. 

\subsection{Signal and Background Models}
\par Both the KMR and CHIDe models are implemented in the Forward Physics Monte Carlo (FPMC).
For signal we generate exclusive \hwwll\ within KMR using FPMC 
and shower with Herwig. The results are then run through the full simulation
of the ATLAS detector. This signal is characterized by very little activity 
around the dileptons. 
\par Background processes can be separated into exclusive and inclusive. 
Inclusive backgrounds are those in which the two protons dissociate and 
the by-products are detected by the ATLAS detector. All these backgrounds 
are estimated using several MC samples except the W+jets background, in 
which the jet is misidentified as a lepton. A 
tested and verified data-driven method is used to estimate W+jets. See 
Section VIC from Ref.~\cite{ATLASCONF2014060} for a detailed description 
of this method. The rest of the backgrounds are listed in Table~\ref{table:background}.

\par The Z/$\gam^*$+jets background is dominated by \zGamtau+jets in 
which the taus decay leptonically to a different flavor
channel. If the jets are misidentified as \MET\ the two leptons from the final state 
would fake the \hwwll\ signature. Of the \zGamtau+jets processes, the 0-jet process
contributes the most signal contamination because it has the least activity
around the dilepton system. These processes are generated with Alpgen and showered with Herwig. 

\par \ttbar\ is a background to the Higgs signature when the two top quarks decay to a WW system
and a pair of b jets. MC@NLO is used to generate this process and Jimmy is used for showering. 
A single top quark is also a background process because a b jet can be misidentified 
as a lepton. Powheg is used to generate single top events and the showering is done 
with Pythia. 

\par Because the signal is produced through gluon exchanges (See  Figure~\ref{fig:exclH}), ggF Higgs
is a background. ggF Higgs is generated with Powheg and showered with Pythia8. Apart from the intensity 
of activity about the dilepton system, the kinematics of ggF Higgs is identical to that of 
the exclusive Higgs.

\par Another irreducible inclusive background is a pair of W bosons. It is refered to as 
{\it inclusive WW} in this note. The final decay products are identical to signal so 
kinematic distributions are used to separate this background from signal. This is modelled 
by Powheg and Pythia.

\par The rest of the inclusive backgrounds will be referred in this note as {\it Other VV}. This is because they
all constitute a pair of vector bosons. These are $W\gamma$, $W\gamma^*$, $Z\gamma$, $Z\gamma^*$, ZZ, WZ and 
double-parton interaction (DPI). The MC samples for these processes are listed in Table~\ref{table:background}. 
     
\par The main contribution to the exclusive background is the exclusive boson pair. 
Exclusive lepton pairs have a very small cross section compared to exclusive boson pairs. Regardless, 
they are also considered in this study. 
In both cases photons are exchanged, in contrast to gluons exchanged 
in the exclusive Higgs production.
Three diagrams contribute to the exclusive boson pairs. Figure~\ref{fig:excWW} shows the three different diagrams.
In a purely elastic process none of the two protons dissociate during the collision. The intact protons 
vanish along the beamline and the fiducial region sees only the WW decay products. In a Single Dissociative (SD) process
one of the protons dissociate, but the remnants of the dissociation 
vanish along the beamline. A Double Dissociative (DD) process is similar to an inclusive process, but the proton remnants disappear along the beamline.
The three processes have identical kinematic process, so it is impossible to 
distinguish them. Only the purely elastic diagram is modeled by the available MC, Herwig++. 
Data driven methods are used to estimate the SD and DD contributions.    

\begin{figure}[!htb]
\minipage{0.32\textwidth}
\begin{fmffile}{fgraphs0D}
\begin{fmfgraph*}(80,60)
\fmfleft{i1,i2}
\fmfright{o1,o2,o3,o4}
\fmf{fermion}{i1,v1}
\fmf{fermion}{v1,o1}
\fmfv{}{o1}
\fmf{fermion}{i2,v2}
\fmf{fermion}{v2,o4}
\fmfv{}{o4}
\fmf{photon,label=$\gamma$}{v1,v3}
\fmf{photon}{v3,o3}
\fmfv{label=$W^+$}{o3}
\fmf{photon,label=$\gamma$}{v2,v3}
\fmf{photon}{v3,o2}
\fmfv{label=$W^-$}{o2}
\fmfdot{v3}
\end{fmfgraph*}
\end{fmffile}
\endminipage\hfill
\minipage{0.32\textwidth}
\begin{fmffile}{fgraphsSD}
\begin{fmfgraph*}(80,60)
\fmfleft{i1,i2}
\fmfright{o1,o2,o3,o4,o5,o6}
\fmf{fermion}{i1,v1}
\fmf{fermion}{v1,o1}
\fmfv{}{o1}
\fmf{fermion}{i2,v2}
\fmfforce{.5w,.8h}{v2}
\fmf{dashes_arrow}{v2,o4}
\fmfv{}{o4}
\fmf{photon,label=$\gamma$}{v1,v3}
\fmf{photon}{v3,o3}
\fmfv{label=$W^+$}{o3}
\fmfforce{w,0.65h}{o3}
\fmf{photon,label=$\gamma$}{v2,v3}
\fmf{photon}{v3,o2}
\fmfv{label=$W^-$}{o2}
\fmfforce{w,0.35h}{o2}
\fmf{dashes_arrow}{v2,o6}
\fmf{dashes_arrow}{v2,o5}
\fmfforce{w,h}{o6}
\fmfforce{w,0.9h}{o5}
\fmfforce{w,0.8h}{o4}
\fmfdot{v3}
\end{fmfgraph*}
\end{fmffile}
\endminipage\hfill
\minipage{0.32\textwidth}%
\begin{fmffile}{fgraphsDD}
\begin{fmfgraph*}(80,60)
\fmfleft{i1,i2}
\fmfright{o1,o2,o3,o4,o5,o6,o7,o8}
\fmf{fermion}{i1,v1}
\fmf{dashes_arrow}{v1,o1}
\fmf{dashes_arrow}{v1,o2}
\fmf{dashes_arrow}{v1,o3}
\fmfforce{.5w,0.15h}{v1}
\fmfforce{w,0}{o1}
\fmfforce{w,0.05h}{o1}
\fmfforce{w,0.1h}{o2}
\fmfforce{w,0.15h}{o3}
\fmf{photon,label=$\gamma$}{v1,v3}
\fmf{photon}{v3,o5}
\fmfv{label=$W^+$}{o5}
\fmfforce{w,0.7h}{o5}
\fmf{fermion}{i2,v2}
\fmfforce{.5w,0.85h}{v2}
\fmf{photon,label=$\gamma$}{v2,v3}
\fmf{photon}{v3,o4}
\fmfv{label=$W^-$}{o4}
\fmfforce{w,0.3h}{o4}
\fmf{dashes_arrow}{v2,o8}
\fmf{dashes_arrow}{v2,o7}
\fmf{dashes_arrow}{v2,o6}
\fmfforce{w,h}{o8}
\fmfforce{w,0.95h}{o7}
\fmfforce{w,0.9h}{o6}
\fmfdot{v3}
\end{fmfgraph*}
\end{fmffile}
\endminipage
\caption{Diagrams that contribute to the exclusive WW processes. [Left] 
A purely elastice process. No proton dissociates and the two protons vanish along the beamline. [Middle]
A single dissociative (SD) process. One proton dissociates but vanishes along the beamline. [Right]
A double dissociative (DD) process. Both protons dissociate but the remnants vanish along the beamline.
Rapidity gaps are present in all three diagrams and the kinematic distributions are identical, so
they are indistinguishable.}
\label{fig:exclWW}
\end{figure}

\begin{table}[!h]
\begin{center}
\begin{tabular}{l|l}
Inclusive Background & Exclusive Background \\
\hline
Drell Yan Z/$\gamma^*$+jets (Alpgen+Herwig) & 															  WW (Herwig++) \\
W$\gamma$ +jets (Alpgen+Jimmy) & $l^{+}l^{-}$ (Herwig++)\\
WZ, WW, ZZ, ggF Higgs (Powheg+Pythia8) 														& \\
\ttbar\ (MC@NLO+Jimmy) & \\
single top (AcerMC+Pythia) & \\
Z$\gamma$ +jets (Sherpa) & \\
\end{tabular}
\caption{Background samples used.}
\label{table:background}
\end{center}
\end{table}

\subsection{Data Samples}
All the data used is from the ATLAS 2012 8~\TeV\ run and sums up to 
a total integrated luminosity of 20.3~\ifb. The D3PDs used are the most up-to-date and are 
identified with the tag p1328/p1329. The integrated luminosity is obtained 
through an ATLAS luminosity calculator~\cite{lumi}. 
