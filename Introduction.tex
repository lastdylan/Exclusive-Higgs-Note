%\clearpage
\section{Introduction}
\label{sec:introduction}

\par Theoretical studies that attempt to
estimate the event rates for the exclusive diffractive Higgs boson at
the LHC have been previously preformed~\cite{Hdijet}~\cite{canLHC}. 
In these studies rapidity gaps between
the Higgs decay products and the outgoing protons play the most active
role. At high luminosities however, these rapidity gaps are corrupted
by pileup events and can hardly be used to separate exclusive processes from 
inclusive processes. This result has diverted attention to ATLAS' ALFA detectors
which could be used as taggers for the outgoing protons. A search for an
exclusive pion pair has been used to test the feasibility of these ALFA 
detectors~\cite{pionPair}. 
 
\par The most recent experimental searches for exclusive events were done
 at 7 TeV by ATLAS and CMS~\cite{CMSmumu}~\cite{CMSee}~\cite{MonteNote}.
Here isolated lepton pairs from QED production were searched for by looking 
for isolated vertices with only two lepton tracks; no rapidity gaps or 
proton tagging was used. At 8 TeV however the strategy of looking for 
vertices with only two tracks was found to be inefficient using exclusive 
\HWWll\ Monte Carlo samples. Feasibility studies by the tracking group confirmed 
this inefficiency and demonstrated that vertexing algorithms at 8 TeV 
tend to generously associate tracks with vertices. This obviously contaminates the exclusive signal.  

\par In this note a search for exclusive events using an 
algorithm that utilizes neither vertexing, rapidity gaps nor proton tagging is presented. 
Exclusive \HWWll\ events survival rate is improved. 
The note is organized as follows: Section~\ref{sec:theory} 
describes briefly the physics of exclusive diffractive Higgs production.
The Khoze Martin Ryskin (KMR) and CHIDe models are described and their 
sources of theoretical systematic uncertainties are outlined. A discussion of 
the Monte Carlo (MC) sample used to study the signal and a list of backgrounds 
and their MC is outlined as well in this section. Section~\ref{sec:selection} 
describes the object and event selection. Section~\ref{sec:exclusivity} 
introduces the algorithm for selecting exclusive events. Here the signal 
survival rate is quoted and the exclusivity cut is tested in data. Section~\ref{sec:results} 
estimates the expected number of signal events versus background events 
in the signal region.   
